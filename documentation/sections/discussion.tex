\section{Diskussion}
Dieser Abschnitt widmet sich der Diskussion von Vor- und Nachteilen, welche sich durch den Gebrauch dieser Visualisierung bzw. dieser Anwendung ergeben. Weiterhin werden hier Designentscheidungen kritisch betrachtet und im Kontext beurteilt.\\
\\
Die Auswahl der einzelnen Visualisierungen ist ein Punkt der sowohl Vor- als auch Nachteile aufweist. Normalerweise sollte eine Visualisierung f\"ur einen bestimmten Anwendungszweck aussagekr\"aftig genug sein und der Nutzer keine zweite Ansicht ben\"otigen. In diesem Projekt werden jedoch zwei Ansichten als Kapitelauswahl angeboten. Ein Punkt der hier jedoch oft nicht beachtet wird ist, dass auch die Benutzergruppen eine Rolle bei der Erstellung solch einer Anwendung spielen. Eine Visualisierung kann nicht auf eine Benutzergruppe abgestimmt werden und dabei allen anderen Nutzern die selben Informationen bieten. Hierbei sind allgemeine Kenntnisse und ein Kontext bezogenes Vorwissen zu ber\"ucksichtigen. Dies ist der Grund, warum die Aufspaltung als notwendig und voranbringend angesehen wird.\\
\\
Auch abseits der Benutzergruppe sind Argumente f\"ur eine Doppelansicht zu finden. Wie in Kapitel \ref{subsec:tree} erw\"ahnt, ist in diesem Projekt f\"ur den Baum lediglich eine Tiefe von maximal vier bis f\"unf Ebenen zu erwarten (Dokument / Root, Kapitel, Unterkapitel, Unterunterkapitel und ggf. Unterunterunterkapitel). Die H\"ohe jedoch wird von der Anzahl an Nachbarkapiteln einer Ebene bestimmt, welche je nach Dokument sehr unterschiedlich ausfallen k\"onnen. In der Vollansicht stellt dies kein Problem dar, wird diese jedoch verkleinert, so k\"onnen einzelne Elemente einer breit gef\"acherten Dokumentstruktur gegebenenfalls nur noch schwer auseinander gehalten werden. Vor allem hier zeigt sich wiederum der Vorteil einer Aufteilung in zwei Visualisierungen. Aufgrund der hohen Data-Density des Sunbursts w\"urde nun diese Ansicht ausgew\"ahlt werden um weitere Navigationen zu vollziehen. Der Baum hingegen besitzt eine geringe Datendichte und ist somit f\"ur solch eine Navigation weniger geeignet. Der Fokus des Baumes liegt jedoch auch mehr in der strukturierten, als in einer zentrierten Darstellung. Dies ist auch einer der Gr\"unde, warum das Sunburst-Diagramm als Hauptansicht und die Baumstruktur vorwiegend als Alternative gew\"ahlt wurde.\\
\\
Ein weiterer m\"oglicher Kritikpunkt ist die Verwendung von Diagrammen, welche im Alltag eher selten auftreten, wie beispielsweise das Sunburst-Diagramm. Die gezielte Verwendung der Visualisierung kann f\"ur bestimmte Nutzergruppen durch eine l\"angere Einarbeitungszeit verz\"ogert werden, sodass diese im schlimmsten Falle eine Einf\"uhrung ben\"otigen k\"onnten. Auf den ersten Blick ist nicht umbedingt f\"ur jeden direkt ersichtlich, wof\"ur die einzelnen Elemente stehen oder wie sie verwendet werden k\"onnen. Eine m\"ogliche Hilfestellung k\"onnte hier durch eine ausf\"uhrlichere Legende geboten werden. Diese stellen jedoch f\"ur gew\"ohnlich unn\"otigen Chartjunk dar, sodass eine aussagekr\"aftige Visualisierung zu bevorzugen ist. Aus diesem Grund wurden Legenden vorwiegend f\"ur die Auflistung von Farbwerten verwendet und weniger zur Erl\"auterung von Interaktionen. Diese Farbwerte visualisieren selber Daten, ohne welche der Nutzen einen weitaus geringeren Informationsgewinn haben w\"urde. Zudem wurde bedacht, dass die Interaktion mit der Visualisierung einen Dominoeffekt aufweist. Auch wenn der Nutzer nicht auf den ersten Blick die Funktionsweise erkennt, so wird diese sp\"atestens durch das Zusammenspiel der einzelnen Elemente deutlich. Auswahlen in einem Diagramm haben direkte Auswirkungen auf die anderen Diagramme, sodass ihre Zusammengeh\"origkeit schnell ersichtlich wird. Einzeln betrachtet bieten die Visualisierungen bereits einige Informationen, zusammen jedoch bieten sie dem Nutzer ein weites Repertoire an M\"oglichkeiten.\\
\\
Unter dem Aspekt der Verst\"andlichkeit sollte auch das Blasen-Diagramm genannt werden. Im Grunde stellt dieses eine Rangliste der gezeigten Begriffe dar. Aufgrund der, in Kapitel \ref{subsec:bubble} beschriebenen, Verzerrungsfaktoren werden zu so einem Zweck meist andere Visualisierungen verwendet, \"uber welche die Gr\"o{\ss}enverh\"altnisse wahrheitsgem\"a{\ss}er interpretiert werden k\"onnen. Somit war auch f\"ur dieses Projekt zun\"achst ein Balkendiagramm angedacht, da dieses eine gute Grundlage f\"ur vergleichbare Elemente bietet. Im Bezug auf den Kontext dieser Visualisierung wurde sich dann schlie{\ss}lich doch gegen den Gebrauch eines Balkendiagramms und f\"ur die Nutzung des Blasen-Diagramms entschieden. Zwei Gr\"unde wurden bereits in Kapitel \ref{subsec:bubble} genannt, sodass die verzerrte Wahrnehmung unterschiedlicher Radien zu einer gewollten Hervorhebung der kritischen Bereiche f\"uhrt und ein roter Faden ersichtlich werden k\"onnte. Einen weiteren Grund stellt hier jedoch die Tatsache dar, das ungewohnte Visualisierung zur Exploration auffordern. \"Ahnlich zu hellen und verspielten Einf\"arbungen wird auch hierdurch der Nutzer indirekt aufgefordert die M\"oglichkeiten der Anwendung auszutesten, sodass der zuvor genannte Dominoeffekt ausgel\"ost wird. Da ein Fokus dieser Anwendung die explorative Erkundung von Dokumenten darstellt, wurde versucht, diesen Aspekt in der kompletten Visualisierung beizubehalten, was zus\"atzlich die Wahl des Sunburst-Diagramms unterst\"utzt.\\
\\
Letztlich bleibt zu erw\"ahnen, dass durch die Nutzung der Anwendung gro{\ss}fl\"achige \"Uberarbeitungen eines Dokumentes in geringer Zeit statt finden k\"onnen. Anstatt diese vollst\"andig und geradlinig zu durchsuchen, k\"onnen diese Problemstellen nun gezielt aufgefunden und beseitigt werden. Durch den reduzierten Bereinigungsaufwand von morphologischen Fehlern, kann nun mehr Zeit f\"ur die inhaltliche \"Uberarbeitung des Dokumenteninhalts aufgebracht werden. Die unterschiedlichen Arten der Visualisierung \"ubertragen diesen Vorteil auch auf Personen, welche als Korrekturleser eingesetzt werden und sich in dem Dokument nicht auskennen. Gerade solche Personen verbringen einen Gro{\ss}teil der Bearbeitungszeit mit dem Ausbessern von redundanten Fehlern, welche dem Schriftsteller bis dato nicht bekannt waren. Da gerade solche durch diese Visualisierung auffindbar sind, k\"onnen auch diese Nutzer ihre Zeit auf inhaltliche Verbesserungen fokussieren, wodurch die Qualit\"at des Textes potentiell ansteigt.\\
\\
Unbedacht genutzt kann dies jedoch auch ein Nachteil sein. Es k\"onnte vorkommen, das sich Korrekturleser ausschlie{\ss}lich auf die so gefundenen Problemstellen konzentrieren und keinen Gesamteindruck erlangen, welcher jedoch f\"ur eine endg\"ultige Beurteilung des Dokuments uner\"lasslich ist. Hier werden keine semantischen Zusammenh\"ange erfasst, sodass ein Text nach der Korrektur zwar syntaktisch fehlerfrei sein k\"onnte, diesem jedoch vollst\"andig der Inhalt fehlt. Zudem besteht ein wissenschaftliches Dokument zu gr{\ss}en Teilen aus dem Zusammenspiel von Bildern und Text. Diese Zusammenh\"ange gehen nur aus einer semantischen Analyse hervor und m\"ussen somit weiterhin manuel durchgef\"uhrt werden.\\