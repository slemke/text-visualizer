\section{Conclusion}
Zum Abschluss l\"asst sich sagen, dass die in der Aufgabendefinition formulierten Ziele gut erreicht worden sind. Durch die Kombination aus Visualisierung und Textdarstellung lassen sich m\"ogliche Probleme durch selbstdefinierte Filter erkennen und bieten einen Mehrwert, welcher beim Verfassen von wissenschaftlichen Arbeiten oder \"ahnlichen Texten Anwendung finden kann. 

Im Verlauf der Projektdurchf\"uhrung gab es einige Punkte die sich nicht komplett umsetzen ließen und hier nochmal aufgegriffen werden.\\

\subsection*{Nominalstil}
In der urspr\"unglichen Aufgabendefinition war auch die Markierung von S\"atzen, welche eine Tendenz zum Nominalstil aufweisen, geplant. Dazu wurden die Token mit entsprechenden Part-of-Speech Tags versehen, welche sich durch den Text Viewer hervorheben lassen. Entsprechende Kennwerte w\"ahren eine gute Erweiterung f\"ur das Projekt gewesen, wurden allerdings aus Zeitgr\"unden und dem notwendigen sprachlichen Wissen nicht weiter verfolgt.

\subsection*{Einlesen von Dokumenten}
Zum aktuellen Zeitpunkt ist es nur m\"oglich, vorher standardisierte Texte in XML einzulesen, zu analysieren und am Ende zu visualisieren. Ein weiteres m\"ogliches Feature w\"are die Implementierung einer automatischen Analyse, welche hochgeladene Dokumente bearbeitet und im Anschluss anzeigt.

\subsection*{Visualisierung des Roten Fadens}
Eine weitere Überlegung war die Umsetzung einer gezielten Visualisierung des Roten Fadens. Dabei sollten mit Hilfe von Part-of-Speech Tagging die Token entsprechend annotiert und gefiltert werden. Ziel war es eine einfache Alternative zum Topic Modelling zu implementieren, bei dem es darum geht, aus dem Text das unbekannte Thema zu finden. Die POS-Tags sollten dabei prim\"ar nach Nomen gefiltert werden, da diese Orte, Personen oder andere Entit\"aten beschreiben.\\
\\
Die H\"aufigkeit dieser annotierten Token sollte als alternative zum Topic Modelling nun die einzelnen Abschnitte des Textes beschreiben. Die H\"aufigkeiten k\"onnten dann in einer Baumstruktur oder einer anderen hierarchischen Visualisierung dargestellt werden und einen m\"oglichen Roten Faden aufzeigen.

\subsection*{Statistik}
Die bereits im Text Mining Kapitel angesprochenen Mittel der deskriptiven Statistik, welche nicht umgesetzt wurden, k\"onnen einen zus\"atzlichen Einblick in das Dokument bieten. Dabei sind einfache Visualisierungen wie Boxplots hilfreich, welche ein Verst\"andnis f\"ur die Verteilung der Werte bieten k\"onnten und so Ausreißer leichter erkennen ließen. Solch eine Visualisierung w\"ahre somit eine geeignete Erweiterung der bestehenden Detailansicht.

\subsection*{Texteditor}
Neben dem Einlesen von Dokumenten k\"onnte auch eine Erweiterung des Text Viewers implementiert werden. So k\"onnten m\"ogliche Probleme live behoben werden bzw. direktes Feedback zu dem geschriebenen Text vermittelt werden. Auf diese Weise w\"urde ein stetiger Arbeitszyklus entstehen, an dessen Ende das fertige Textdokument steht. 

\subsection*{Mehrfachauswahl in Visualisierung}
Ein weiterer, wichtiger Punkt wird durch die M\"oglichkeit abgebildet, die verschiedenen Inhalte zu vergleichen. Aktuell lassen sich zwar Kapitel und deren Unterkapitel ausw\"ahlen, allerdings k\"onnen keine zwei (oder mehr) Hauptkapitel gleichzeitig angewählt werden. Durch eine Mehrfachauswahl und der entsprechenden Berechnung der Daten lassen sich m\"oglicherweise Zusammenh\"ange zwischen einzelnen Kapiteln darstellen, welche wiederum auf einen roten Faden verweisen k\"onnten.

\subsection*{Zusätzliche Filteroptionen}
Zus\"atzliche Filteroptionen k\"onnen eine noch tiefere Exploration der Daten erm\"oglichen. Neben einfachen Filtern wie das aktivieren bzw. deaktivieren von F\"ullw\"ortern oder Worth\"aufigkeiten, kann auch ein Filtern nach POS-Tags einen anderen Einblick erm\"oglichen.

\subsection*{Wichtige Erg\"anzungen zu dem Projekt}
Da die Zeit begrenzt war und wir uns vorwiegend auf eine Funktionst\"uchtige Visualisierung konzentriert haben, konnte dem Text Mining leider zu wenig Zeit zugeordnet werden. Aus diesem Grund befinden sich einige kleine Fehler innerhalb der Datenquellen, welche nun anhand der fertigen Visualisierung ersichtlich wurden. So befinden sich in Oberkapiteln bei der Suche nach Redundanzen alle Begriffe der Kindknoten, jedoch keine zusammengefasste Version. Diese tauchen somit in dem Blasen-Diagramm redundant auf und besitzen weiterhin ihre Werte im Verh\"altnis zu den Kindknoten. Weiterhin beinhalten die Kapitel Abstract, Introduction, Discussion and future work und conclusion abgesehen von den Redundanzen innerhalb der Kapitelauswahl keinen Wert. Auch dies liegt an den unvollst\"andigen Daten. In keiner dieser F\"allen findet eine Verf\"alschung durch die Visualisierung statt, was anhand der untersten Kindknoten jedes Kapitels ersichtlich ist. Abgesehen von diesen Werten ist der Datenbestand jedoch vollst\"andig und bereit exploriert zu werden.