%% The ``\maketitle'' command must be the first command after the
%% ``\begin{document}'' command. It prepares and prints the title block.

%% the only exception to this rule is the \firstsection command
\firstsection{Einleitung}

\maketitle

%% Introduction starts here %%
In den letzten Jahren haben immer mehr digitale Medien Einzug in das allt\"agliche Leben gehalten. Insbesondere das Internet ist dabei f\"ur einen Großteil der Ver\"anderung verantwortlich. Doch neben Bildern, Videos und Audio ist der Text, gerade auch durch das Internet, immer noch das wichtigste Kommunikationsmittel f\"ur Wissen.\\
\\
Doch nicht nur im allt\"aglichen Leben, insbesondere im wissenschaftlichen Umfeld ist Schrift und Sprache wichtig. Das Verfassen von Hausarbeiten oder das sp\"atere Schreiben von Artikeln f\"ur Journale erfordert dabei ein gewisses Maß an Qualit\"at. Diese Qualit\"at wird oft durch Erfahrung, einhalten von diversen Richtlinien und Zeit in Form von wiederholten Iterationen erreicht.\\
\\
Dieses Projekt versucht diesen Prozess mit Hilfe von Visualisierungen von Text Mining Daten zu unterst\"utzen. In vielen Leitf\"aden zu wissenschaftlichem Schreiben finden sich Richtlinien wie Pr\"azision oder das vermeiden von Wortwiederholungen. Diese Werte lassen sich mit Text Mining und Mitteln der deskriptiven Statistik leicht ermitteln und dienen als Grundlage f\"ur die, in diesem Dokument beschriebenen Visualisierungen. Aus diesem Grund sollten die Texte auf folgende Eigenschaften untersucht werden:\\
\begin{itemize}
\item Worth\"aufigkeit (Redundanz)
\item Satzl\"ange (Komplexit\"at)
\item Satzzeichen (Komplexit\"at)
\item F\"ullw\"orter (Pr\"azision)
\end{itemize}
Im folgenden Kapitel wird beschrieben, welche Text Mining Verfahren eingesetzt wurden, um die Daten für die Visualisierung der oben beschrieben Eigenschaften zu gewinnen.
