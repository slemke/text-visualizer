%% The ``\maketitle'' command must be the first command after the
%% ``\begin{document}'' command. It prepares and prints the title block.

%% the only exception to this rule is the \firstsection command
\firstsection{Einleitung}

\maketitle

%% Introduction starts here %%
In den letzten Jahren haben immer mehr digitale Medien Einzug in das alltägliche Leben gehalten. Insbesondere das Internet ist dabei für einen Großteil der Veränderung verantwortlich. Doch neben Bildern, Videos und Audio ist der Text, gerade auch durch das Internet, immer noch das wichtigste Kommunikationsmittel für Wissen.\\

Doch nicht nur im alltäglichen Leben, insbesondere im wissenschaftlichen Umfeld ist Schrift und Sprache wichtig. Das Verfassen von Hausarbeiten oder das spätere Schreiben von Artikeln für Journale erfordert dabei ein gewisses Maß an Qualität. Diese Qualität wird oft durch Erfahrung, einhalten von diversen Richtlinen und Zeit in Form von wiederholten Iterationen erreicht.\\

Dieses Projekt versucht diesen Prozess mit Hilfe von Visualisierungen von Text Mining Daten zu unterstützen. In vielen Leitfäden zu wissenschaftlichem Schreiben finden sich Richtlinien wie Präzision oder das vermeiden von Wortwiederholungen. Diese Werte lassen sich mit Text Mining und Mitteln der deskriptiven Statistik leicht ermitteln und dienen als Grundlage für die in diesem Dokument beschriebenen Visualisierungen. Aus diesem Grund sollten die Texte auf folgende Eigenschaften untersucht werden:\\

\begin{itemize}
\item Worthäufigkeit (Redundanz)
\item Satzlänge (Komplexität)
\item Satzzeichen (Komplexität)
\item Füllwörter (Präzision)
\end{itemize}

Im folgenden Kapitel wird beschrieben, welche Text Mining Verfahren eingesetzt wurden, um die Daten für die Visualisierung der oben beschrieben Eigenschaften zu gewinnen. 