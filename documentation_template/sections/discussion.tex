\section{Diskussion}
Dieser Abschnitt widmet sich der Diskussion von Vor- und Nachteilen, welche durch den Gebrauch dieser Visualisierung bzw. dieser Anwendung ergeben.

\subsection{Unterst\"utzung bei der Suche nach syntaktischen Problemstellen}
Durch die Nutzung der Anwendung k\"onnen gro{\ss}fl\"achige \"Uberarbeitungen eines Dokumentes in geringer Zeit statt finden. Anstatt diese vollst\"andig und geradlinig zu durchsuchen, k\"onnen diese Problemstellen nun gezielt aufgefunden und beseitigt werden. Durch den reduzierten Bereinigungsaufwand von syntaktischen Fehlern, kann nun mehr Zeit f\"ur die inhaltliche \"Uberarbeitung des Dokumenteninhalts aufgebracht werden. Die unterschiedlichen Arten der Visualisierung \"ubertragen diesen Vorteil auch auf Personen, welche als Korrekturleser eingesetzt werden und sich in dem Dokument nicht auskennen. Gerade solche Personen verbringen einen Gro{\ss}teil der Bearbeitungszeit mit dem Ausbessern von redundanten Fehlern, welche dem Schriftsteller bis dato nicht bekannt waren. Da gerade solche durch diese Visualisierung auffindbar sind, k\"onnen auch diese ihre Zeit auf inhaltliche Verbesserungen fokussieren, wodurch die Qualit\"at des Textes potentiell ansteigt.\\
Unbedacht genutzt kann dies jedoch auch ein Nachteil sein. Es k\"onnte vorkommen, das sich Korrekturleser ausschlie{\ss}lich auf die so gefunden Problemstellen konzentrieren und keinen Gesamteindruck erlangen, welcher jedoch f\"ur eine endg\"ultige Beurteilung des Dokuments uner\"lasslich ist. Hier werden keine semantischen Zusammenh\"ange erfasst, sodass ein Text nach der Korrektur zwar syntaktisch fehlerfrei sein k\"onnte, diesem jedoch vollst\"andig der Inhalt fehlt. Zudem besteht ein wissenschaftliches Dokument zu gr{\ss}en Teilen aus dem Zusammenspiel von Bildern und Text. 

Das Dokument muss hierf\"ur also immernoch gelesen werden

Immernoch eingew\"ohnungszeit da Diagramme verwendet wurden, welche im Alltag eher selten auftreten.