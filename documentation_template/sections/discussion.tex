\section{Diskussion}
Dieser Abschnitt widmet sich der Diskussion von Vor- und Nachteilen, welche sich durch den Gebrauch dieser Visualisierung bzw. dieser Anwendung ergeben. Weiterhin werden hier Designentscheidungen kritisch betrachtet und im Kontext beurteilt.\\
\\
Die Auswahl der einzelnen Visualisierungen ist ein weiterer Punkt der sowohl Vor- als auch Nachteile aufweist. Normalerweise sollte eine Visualisierung f\"ur einen bestimmten Anwendungszweck aussagekr\"aftig genug sein und der Nutzer keine zweite Ansicht ben\"otigen. In diesem Projekt werden jedoch zwei Ansichten als Kapitelauswahl angeboten. Ein Punkt der hier jedoch oft nicht beachtet wird ist, dass auch die Benutzergruppen eine Rolle bei der Erstellung solch einer Anwendung spielen. Eine Visualisierung kann nicht auf eine Benutzergruppe abgestimmt werden und dabei allen anderen Nutzern die selben Informationen bieten. Hierbei sind allgemeine Kenntnisse und ein Kontext bezogenes Vorwissen zu ber\"ucksichtigen. Dies ist der Grund, warum die Aufspaltung als notwendig und voranbringend angesehen wird.\\
\\
Auch abseits der Benutzergruppe sind Argumente f\"ur eine Doppeltansicht zu finden. Wie in Kapitel \ref{subsec:tree} erw\"ahnt, ist in diesem Projekt f\"ur den Baum lediglich eine Tiefe von maximal vier bis f\"unf Ebenen zu erwarten (Dokument / Root, Kapitel, Unterkapitel, Unterunterkapitel und ggf. Unterunterunterkapitel). Die H\"ohe jedoch wird von der Anzahl an Nachbarkapiteln einer Ebene bestimmt, welche je nach Dokument sehr unterschiedlich ausfallen k\"onnen. In der Vollansicht stellt dies kein Problem dar, wird diese jedoch verkleinert, so k\"onnen einzelne Elemente einer breit gef\"acherten Dokumentstruktur gegebenenfalls nur noch schwer auseinander gehalten werden. Vor allem hier zeigt sich wiederum der Vorteil einer Aufteilung in zwei Visualisierungen. Aufgrund der hohen Data-Density des Sunbursts w\"urde nun diese Ansicht ausgew\"ahlt werden um weitere Navigationen zu vollziehen. Der Baum hingegen besitzt eine geringe Datendichte und ist somit f\"ur solch eine Navigation weniger geeignet. Der Fokus des Baumes liegt jedoch auch mehr in der strukturierten, als in einer zentrierten Darstellung. Dies ist auch einer der Gr\"unde, warum das Sunburst-Diagramm als Hauptansicht und die Baumstruktur lediglich als Erweiterung gew\"ahlt wurde.\\
\\

- Immernoch eingew\"ohnungszeit da Diagramme verwendet wurden, welche im Alltag eher selten auftreten. Auf den ersten Blick ist nicht umbedingt f\"ur jeden direkt ersichtlich wof\"ur die einzelnen Elemente stehen oder wie sie verwendet werden k\"onnen. Zudem bilden legenden f\"ur gew\"ohnlich unn\"otigen Chartjunk und sollten umgangen werden. Hier wurden sie jedoch vorwiegend f\"ur farbwerte verwendet, welche selber Daten visualisieren und ohne welche der Nutzen weitaus weniger Informationsgewinn haben w\"urde -> durch das Zusammenspiel wird dieses jedoch klar. Eine Auswahl in einem Diagramm hat eine direkte Auswirkung auf die anderen Diagramme, sodass ihre Zusammengeh\"origkeit schnell ersichtlich wird. Einzeln betrachtet bieten die Visualisierungen bereits einige Informationen, zusammen jedoch bieten sie dem Nutzer ein weites Repertoir an M\"oglichkeiten.

Letztlich bleibt zu erw\"ahnen, dass durch die Nutzung der Anwendung gro{\ss}fl\"achige \"Uberarbeitungen eines Dokumentes in geringer Zeit statt finden k\"onnen. Anstatt diese vollst\"andig und geradlinig zu durchsuchen, k\"onnen diese Problemstellen nun gezielt aufgefunden und beseitigt werden. Durch den reduzierten Bereinigungsaufwand von syntaktischen Fehlern, kann nun mehr Zeit f\"ur die inhaltliche \"Uberarbeitung des Dokumenteninhalts aufgebracht werden. Die unterschiedlichen Arten der Visualisierung \"ubertragen diesen Vorteil auch auf Personen, welche als Korrekturleser eingesetzt werden und sich in dem Dokument nicht auskennen. Gerade solche Personen verbringen einen Gro{\ss}teil der Bearbeitungszeit mit dem Ausbessern von redundanten Fehlern, welche dem Schriftsteller bis dato nicht bekannt waren. Da gerade solche durch diese Visualisierung auffindbar sind, k\"onnen auch diese ihre Zeit auf inhaltliche Verbesserungen fokussieren, wodurch die Qualit\"at des Textes potentiell ansteigt.\\
\\
Unbedacht genutzt kann dies jedoch auch ein Nachteil sein. Es k\"onnte vorkommen, das sich Korrekturleser ausschlie{\ss}lich auf die so gefunden Problemstellen konzentrieren und keinen Gesamteindruck erlangen, welcher jedoch f\"ur eine endg\"ultige Beurteilung des Dokuments uner\"lasslich ist. Hier werden keine semantischen Zusammenh\"ange erfasst, sodass ein Text nach der Korrektur zwar syntaktisch fehlerfrei sein k\"onnte, diesem jedoch vollst\"andig der Inhalt fehlt. Zudem besteht ein wissenschaftliches Dokument zu gr{\ss}en Teilen aus dem Zusammenspiel von Bildern und Text. Diese Zusammenh\"ange gehen nur aus einer semantischen Analyse hervor und m\"ussen somit weiterhin manuel durchgef\"uhrt werden.\\
\\

-> tree / bar chart nennen